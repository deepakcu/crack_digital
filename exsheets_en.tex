% arara: pdflatex
% !arara: biber
% !arara: pdflatex
% arara: pdflatex
% --------------------------------------------------------------------------
% the EXSHEETS package
% 
%   Yet another package for the creation of exercise sheets
% 
% --------------------------------------------------------------------------
% Clemens Niederberger
% Web:    https://bitbucket.org/cgnieder/exsheets/
% E-Mail: contact@mychemistry.eu
% --------------------------------------------------------------------------
% Copyright 2011-2013 Clemens Niederberger
% 
% This work may be distributed and/or modified under the
% conditions of the LaTeX Project Public License, either version 1.3
% of this license or (at your option) any later version.
% The latest version of this license is in
%   http://www.latex-project.org/lppl.txt
% and version 1.3 or later is part of all distributions of LaTeX
% version 2005/12/01 or later.
% 
% This work has the LPPL maintenance status `maintained'.
% 
% The Current Maintainer of this work is Clemens Niederberger.
% --------------------------------------------------------------------------
% The exsheets bundle consists of the files
%  - exsheets.sty, exsheets_headings.def, exsheets_headings.cfg,
%    exsheets_configurations.cfg,
%    exsheets_en.tex, exsheets_en.pdf
%
%  - tasks.sty, tasks.cfg, tasks_en.tex, task_en.pdf
%
%  - cntformats.sty, cntformats_en.tex, cntformats_en.pdf
%
%  - README
% --------------------------------------------------------------------------
% If you have any ideas, questions, suggestions or bugs to report, please
% feel free to contact me.
% --------------------------------------------------------------------------
%
% if you want to compile this documentation you'll need the document class
% `cnpkgdoc' which you can get here:
%    https://bitbucket.org/cgnieder/cnpkgdoc/
% the class is licensed LPPL 1.3 or later
%
% use `pdflatex' and `biber' for compilation
%
\PassOptionsToPackage{supstfm=libertinesups}{superiors}
\documentclass[DIV10,toc=index,toc=bib,numbers=noendperiod]{cnpkgdoc}
% ----------------------------------------------------------------------------
% document layout and typographic features
\docsetup{
  pkg      = {[load-headings,load-tasks]exsheets} ,
  url      = {https://bitbucket.org/cgnieder/exsheets/} ,
  title    = the ExSheets bundle ,
  subtitle = {%
    the packages \ExSheets, \href{tasks_en.pdf}{\Tasks} and
    \href{cntformats_en.pdf}{\cntformats}\\
    \emph{or}\\
    Yet another package for the creation of exercise sheets and exams.%
  } ,
  name     = ExSheets ,
  language = en ,
  modules  = true ,
  code-box = {
    backgroundcolor  = gray!7!white ,
    skipbelow        = .6\baselineskip plus .5ex minus .5ex ,
    skipabove        = .6\baselineskip plus .5ex minus .5ex ,
    innerleftmargin  = 1.5em ,
    innerrightmargin = 2.5em
  } ,
  gobble   = 1
}
\makeatletter
\edef\cntfmtsversion{\@cntfmts@version}
\edef\tasksversion{v\@tasks@version}
\makeatother

\usepackage{libertinehologopatch}

\cnpkgusecolorscheme{friendly}

\usepackage[biblatex]{embrac}[2012/06/29]
  \ChangeEmph{[}[,.02em]{]}[.055em,-.08em]
  \ChangeEmph{(}[-.01em,.04em]{)}[.04em,-.05em]
\usepackage{microtype}
\usepackage[multiple]{fnpct}

\renewcommand*\othersectionlevelsformat[3]{\textcolor{main}{#3\autodot}\enskip}
\renewcommand*\partformat{\textcolor{main}{\partname~\thepart\autodot}}

\pagestyle{headings}

\setcapindent{1.5em}
\setkomafont{caption}{\normalfont\footnotesize\sffamily}
\setkomafont{captionlabel}{\normalfont\footnotesize\sffamily\scshape}

\usepackage{booktabs,array,ragged2e}

% ----------------------------------------------------------------------------
% code examples
% ExSheets:
\addcmds{
  acs, addpoints,
  blank, bigstar, bottomrule,
  checkedchoicebox, choice, choicebox, citetitle, cmd, color, correct,
  DebugExSheets, DeclareInstance, DeclareQuestionClass, DeclareQuestionProperty,
    DeclareTemplateInterface,
  endspacing, examspace, ForEachQuestion,
  GetQuestionProperty, grade,
  includequestions, iflastquestion,
  lastvariant, midrule, NewQuSolPair, numberofquestions,
  points, PrintIfIncludeActiveF, PrintIfIncludeActiveT, PrintIfIncludeActiveTF,
    printsolutions, PrintSolutionsF, PrintSolutionsTF, PrintSolutionsT,
  questionsincludedlast, QuestionNumber, RenewQuSolPair, rightarrow, rlap,
  s, sample, setlength, SetQuestionProperties, SetupExSheets, SetVariations,
    spacing, square, startnewitemline, sumpoints,
  tabcolsep, textcite, textcolor, toprule, totalpoints,
  variant, vary
}

% cntformats:
\newcommand*\cntformats{{\scshape\textcolor{main}{cntformats}}\xspace}
\addcmds{
  @cntfmts@parsed@pattern, AddCounterPattern, eSaveCounterPattern,
  NewCounterPattern, ReadCounterFrom, ReadCounterPattern,
  ReadCounterPatternFrom, SaveCounterPattern, tmpa, tmpb
}

% tasks:
\newcommand*\Tasks{{\scshape\textcolor{main}{tasks}}\xspace}
\addcmds{
  leftthumbsup, NewTasks, settasks, task
}
% ----------------------------------------------------------------------------
% other packages, bibliography, index
\usepackage{xcoffins,wasysym,enumitem,booktabs,siunitx}
\usepackage[accsupp]{acro}
\DeclareAcronym{id}{
  short     = id ,
  long      = Identifier ,
  format    = \scshape ,
  pdfstring = ID ,
  accsupp   = ID
}
\DeclareAcronym{ctan}{
  short     = ctan ,
  long      = Comprehensive \TeX\ Archive Network ,
  format    = \scshape ,
  pdfstring = CTAN ,
  accsupp   = CTAN
}

\usepackage{filecontents}
\usepackage{imakeidx}
\begin{filecontents*}{\jobname.ist}
 heading_prefix "{\\bfseries "
 heading_suffix "\\hfil}\\nopagebreak\n"
 headings_flag  1
 delim_0 "\\dotfill\\hyperpage{"
 delim_1 "\\dotfill\\hyperpage{"
 delim_2 "\\dotfill\\hyperpage{"
 delim_r "}\\textendash\\hyperpage{"
 delim_t "}"
 suffix_2p "\\nohyperpage{\\,f.}"
 suffix_3p "\\nohyperpage{\\,ff.}"
\end{filecontents*}
\indexsetup{othercode=\footnotesize}
\makeindex[options={-s \jobname.ist},intoc,columns=3]

\usepackage{csquotes}
\usepackage[backend=biber,style=alphabetic]{biblatex}
\addbibresource{biblatex-examples.bib}
\addbibresource{\jobname.bib}

\begin{filecontents}{\jobname.bib}
@package{answers,
  title      = {answers},
  author     = {Mike Piff},
  maintainer = {Joseph Wright},
  date       = {2010-10-11},
  version    = {2.13},
  url        = {http://mirror.ctan.org/macros/latex/contrib/answers/}
}
@package{enumitem,
  title      = {enumitem},
  author     = {Javier Bezos},
  date       = {2011-09-28},
  version    = {3.5.2},
  url        = {http://mirror.ctan.org/macros/latex/contrib/enumitem/}
}
@package{eqexam,
  title      = {eqexam},
  author     = {D. P. Story},
  date       = {2011-09-01},
  version    = {3.0k},
  url        = {http://mirror.ctan.org/macros/latex/contrib/eqexam/}
}
@package{esami,
  title      = {esami},
  author     = {Grazia Messineo and Salvatore Vassallo},
  date       = {2013-03-08},
  version    = {1.0},
  url        = {http://mirror.ctan.org/macros/latex/contrib/esami/}
}
@package{exam,
  title      = {exam},
  author     = {Philip Hirschhorn},
  date       = {2011-05-22},
  version    = {2.4},
  url        = {http://mirror.ctan.org/macros/latex/contrib/exam/}
}
@package{examdesign,
  title      = {examdesign},
  author     = {Jason Alexander},
  date       = {2001-03-26},
  version    = {1.1},
  url        = {http://mirror.ctan.org/macros/latex/contrib/examdesign/}
}
@package{exercise,
  title      = {exercise},
  author     = {Paul Pichaureau},
  date       = {2012-05-08},
  version    = {1.58},
  url        = {http://mirror.ctan.org/macros/latex/contrib/exercise/}
}
@package{exsol,
  title      = {exsol},
  author     = {Walter Daems},
  date       = {2013-05-12},
  version    = {0.6},
  url        = {http://mirror.ctan.org/macros/latex/contrib/exsol/}
}
@package{multienum,
  title      = {multienum},
  author     = {Dennis Kletzing},
  date       = {2005-05-19},
  version    = {n.n.},
  url        = {http://mirror.ctan.org/macros/latex/contrib/multienum/}
}
@package{probsoln,
  title      = {probsoln},
  author     = {Nicola L. C. Talbot},
  date       = {2012-08-23},
  version    = {3.04},
  url        = {http://mirror.ctan.org/macros/latex/contrib/probsoln/}
}
\end{filecontents}

% ----------------------------------------------------------------------------
% example definitions that have to be done in the preamble:
\DeclareQuestionClass{difficulty}{difficulties}
\DeclareQuestionProperty{notes}
\DeclareQuestionProperty{reference}
\DeclareQuestionProperty{topic}

\DeclareRelGrades{
  1     = 1 ,
  {1,5} = .9167 ,
  2     = .8333 ,
  {2,5} = .75 ,
  3     = .6667 ,
  {3,5} = .5833 ,
  4     = .5
}

\usepackage{amssymb}
\let\checkmark\relax
\usepackage{dingbat}

\NewQuSolPair
  {question*}[name=\protect\llap{$\bigstar$\space}Bonus Question]
  {solution*}[name=\protect\llap{$\bigstar$\space}Solution]

\NewTasks[style=multiplechoice]{choices}[\choice](3)
\newcommand*\correct{\PrintSolutionsTF{\checkedchoicebox}{\choicebox}}

\usepackage{alphalph}
\NewPatternFormat{aa}{\alphalph}
\NewCounterPattern{testa}{ta}

\usepackage{etoolbox}
\robustify\code

% ----------------------------------------------------------------------------
\newcommand*\required{\hfill\llap{required}\newline}
\newcommand*\optional{\hfill\llap{optional}\newline}

\newcommand*\nooption[2]{\item[\code{\textcolor{key}{#1} = #2}] \cnpkgdocarrow\ }

\newcommand*\unexpsign{$\RHD$}
\newcommand*\expsign{\textcolor{red}{$\rhd$}}
\renewcommand*\cnpkgdoctriangle{\unexpsign}
\newcommand*\expandable{%
  \def\cnpkgdoctriangle{%
    \expsign
    \gdef\cnpkgdoctriangle{\unexpsign}}}

\pdfstringdefDisableCommands{\def\llap#1{#1\space}\def\bigstar{*}}

% ----------------------------------------------------------------------------
% title page
\TitlePicture{%
  \parbox{.8\linewidth}{%
    \ExSheets provides means to create exercises or questions and their
    corresponding solutions.  The questions can be divided into classes and
    can be printed selectively.  Meta-data to questions can be added and
    recovered.
    \par
    The solutions may be printed where they are, can be collected and printed
    at a later point in the document alltogether or section-wise or
    selectively by \acs{id}.\par
    \ExSheets provides a comprehensive interface for styling the headings of
    questions and solutions.}}

\begin{document}

\part{Preliminary}\secidx{Preliminary}
\section{Licence and Requirements}
\secidx[Licence]{Preliminary}\secidx[Requirements]{Preliminary}
\ExSheets is placed under the terms of the \LaTeX{} Project Public License,
version 1.3 or later (\url{http://www.latex-project.org/lppl.txt}).  It has
the status ``maintained.''

\ExSheets loads and needs the following packages: \paket*{l3kernel},
\paket{xparse}, \paket{xtemplate}, \paket{l3sort}, \paket{l3keys2e},
\paket{xcolor}, \paket{ulem}, \paket{etoolbox}, \paket{environ}, and
\paket{pgfcore}.  \ExSheets calls \cmd{normalem} (from the \paket{ulem}
package).

\section{Motivation}\secidx[Motivation]{Preliminary}
There are already quite a number of packages that allow the creation of
exercise sheets or written exams.  Just to name the most common ones:
\paket*{eqexam}~\cite{eqexam}, \paket*{exam}~\cite{exam},
\paket*{examdesign}~\cite{examdesign}, \paket*{exercise}~\cite{exercise},
\paket*{probsoln}~\cite{probsoln}, \paket*{answers}~\cite{answers},
\paket*{esami}~\cite{esami}, \paket*{exsol}~\cite{exsol} (and many
more\footnote{see
  \url{http://www.ctan.org/characterization/primary/document-types/exams-exercise-sets-and-answers/}}).

One thing I missed in all packages that I've tried out\footnote{Well, probably
  I didn't try hard enough\ldots} was a high flexibility in choosing which
questions and solutions should be printed, where which solutions should be
printed and so on, combined with the possibility to assign questions to
different classes so one could for example create two versions of an exam out
of the box.  And --~I can't get enough~-- I also want to be able to use/design
different layouts for questions additional to a standard section-like format.
All these points are realized in \ExSheets.

Additionally one should be able to assign some sort of meta-data to questions
that of course should be easily reusable.  How this can be done is explained
in section~\ref{sec:additional_info}.

Then there is --~at least in Germany~-- the habit of having lists of exercises
aligned in columns but counting from the left to the right instead from up to
down.  \ExSheets provides a possibility for that (see part~\ref{part:tasks}).
I am not quite content with it as it works now, though\footnote{There are
  still other possibilities, for example take a look here:
  \url{http://tex.stackexchange.com/questions/67966/enumerate-in-multicols} or
  at the \paket*{multienum} package~\cite{multienum}.}.

On the other hand \ExSheets doesn't --~and probably won't~-- offer a real
possibility for creating multiple choice questions.  As a fact it doesn't
provide many (if any) means to specify the \emph{type} of question or the
structure.  If these are your needs take a look at
\paket{examdesign}~\cite{examdesign}, for example.  Or exploit the
possibilities \paket{enumitem}~\cite{enumitem} gives you.

I had the idea for this package in 2008.  Back then my \TeX{} skills were by
far not good enough to write it.  Actually, even today I wouldn't have been
able to realize it without all the l3 packages like \paket*{l3kernel} and
\paket*{l3packages}.  I actively began to develop \ExSheets in spring~2011 but
it wasn't until now (September~2012) that I consider it stable enough for
wider usage.  At the time of writing (\today) there still are probably lots of
rough edges let alone bugs so I am very interested in all kinds of feedback.

\section{Additional Packages}
\ExSheets actually bundles three packages: \ExSheets, \Tasks\ and \cntformats.
\Tasks is described in part~\ref{part:tasks} and \cntformats in
part~\ref{part:cntformats}.  These packages provide functionality that is used
by \ExSheets.  They can, however, be used independently from \ExSheets.

The package \cntformats\ is probably not really useful on a user-level but
maybe for package writers.

\changedversion{0.9i}\ExSheets\ used to bundle the \paket*{translations}
package, too, but doesn't any more.  You can find the \paket*{translations}
package as a package of it's own on the \ac{ctan}.

\section{Installation and Documentation}\secidx[Installation]{Preliminary}
If you install \ExSheets manually beware to put the files
\begin{itemize}
  \item[]\verb+exsheets_headings.def+
  \item[]\verb+exsheets_headings.cfg+
\end{itemize}
in the same directory as the \code{exsheets.sty} file\footnote{That is, a
  directory like \code{texmf-local/tex/latex/exsheets}, probably}.  You
\emph{can} install the other packages, \Tasks and \cntformats, in different
locations but since they belong to \ExSheets they probably should be placed in
the same directory.

You should put the file \code{tasks.cfg} in the same directory as the
\code{tasks.sty} file.

As with every manual package installation you need to make sure to put the
files in a directory where \TeX\ can find them and afterwards update the
database.

\section{News}
With version~0.7 there has been a potentially breaking change: the
\code{tasks} environment previously provided by \ExSheets has been extracted
into a package of its own.  This does not change any syntax \emph{per
  se}. However, if you used custom settings then you'll probably run into some
problems.  The options for the environment are no longer set with
\cmd{SetupExSheets} but with \cmd{settasks}.  Also the object that is used for
the list template and its instances has been renamed from
\code{exsheets-tasks} into \code{tasks}.

What's probably even more of a breaking change is a syntax difference of the
\code{tasks} environment: the optional argument for the number of columns is
\emph{no longer set in braces but parentheses}.  This is deliberate as it
reflects the optional nature of the argument better and is consistent with the
syntax of \cmd{NewTasks}, too.

Additionally the labels of the list got an additional offset of \code{1ex}
from the items which will lead to slightly different output.  In some cases
this might actually lead to the most annoying changes.  In this case say
\cmd{settasks}{label-offset=0pt} which should cure things again.

I am very sorry for any inconvenience!  I am trying to keep such changes as
minimal and rare as possibly.  However, it is not always avoidable when a
package is new and still a child. It will grow up eventually.

\ExSheets' other packages -- \href{tasks_en.pdf}{\Tasks} and
\href{cntformats_en.pdf}{\cntformats} -- have gotten their own documentation
which are essentially extracted from this very document you're reading now.
They still will be documented in the main documentation (this document),
though.  Right now (April~25th~2013) the separate documentations do not
contain much more information than is written here but they might, in time,
get more details about the corresponding package.

\section{Thanks}
I need to thank the many users who gave me feedback so far!  For one thing
this shows me that \ExSheets is useful to people.  It also led to many
improvements like new features and countless bug fixes.
\secidx*{Preliminary}

\part{The \ExSheets package}\label{part:exsheets}\secidx{\ExSheets}[ExSheets]
\section{Setup}\secidx{Setup}
The \ExSheets package has three different types of options, kind of.  The
first type are the classic package options which are used when you load
\ExSheets:
\begin{beispiel}[code only]
 \usepackage[<options>]{exsheets}
\end{beispiel}
These options are described in section~\ref{sec:options}.

The second type are options that belong to a specific environment or command.
These options are either used directly with the environment/command
\begin{beispiel}[code only]
 \begin{env}[<options>]
  ...
 \end{env}
\end{beispiel}
or can be set with the setup command:
\begin{beschreibung}
 \Befehl{SetupExSheets}[<module>]{<options>}
\end{beschreibung}
The options of the second type all belong to
\textcolor{module}{\code{modules}}.  Let's say you want to specify some
options of the \code{question} environment.  You can then say the following:
\begin{beispiel}[code only]
 \SetupExSheets[question]{option1,option2=value2}
 % or:
 \SetupExSheets{question/option1,question/option2=value2}
\end{beispiel}
The \textcolor{module}{\code{module}} an option belongs to is written in the
left margin next to the when the option is described.

The third type aren't options at all, actually.  However, thanks to the great
\paket{xtemplate} package you are able to define your own instances of some of
the objects used by \ExSheets.  This is explained in a little more detail in
part~\ref{part:style} on page~\pageref{part:style}\,ff.  This third type,
however, brings in a possible instability: the \paket{xtemplate} package is in
an experimental and developping state.  This means that the sytax of the
package may and possibly will change sometime in the future.  I cannot foresee
what any consequences of that will be for \ExSheets.
\secidx*{Setup}

\section{Package Options}\label{sec:options}\secidx{Package Options}
The package \ExSheets has some options, namely the following ones:
\begin{beschreibung}
  %% color
  \Option{color}{<colour>}\Default{exsheetsblue}
    A custom colour that is used in certain but very rare circumstances.
  %% counter-format
  \Option{counter-format}{<counter-format>}\Default{qu.}
    Formatting of the counter of the questions.  This option takes a special
    kind of string that is described in section~\ref{ssec:counter}.
  \Option{counter-within}{<counter>}\Default{}
    Resets the \code{question} counter with every step of \code{<counter>}.
  %% auto-label
  \Option{auto-label}{\default{true}|false}\Default{false}
    If set to true \ExSheets will automatically place a \cmd*{label}{qu:<id>}
    for each question.  It will also create the question properties \code{ref}
    and \code{pageref}, see section~\ref{sec:additional_info} for more on
    this.
  %% headings
  \Option{headings}{<instance>}\Default{block}
    Choose the style of the questions' and solutions' headings.  There are two
    predefined styles: \code{block} and \code{runin}.
  %% headings-format
  \Option{headings-format}{<code>}\Default{\cmd{normalsize}\cmd{bfseries}}
    This code is placed immediately before the headings of the questions and
    solutions.
  %% load-headings
  \Option{load-headings}{\default{true}|false}\Default{false}
    Loads additional styles for the headings.  More on this is described in
    section~\ref{sec:exsheets-headings}.
  %% load-tasks
  \Option{load-tasks}{\default{true}|false}\Default{false}
    Loads additional styles for the \code{tasks} environment.  See
    part~\ref{part:tasks}.
  %% totoc
  \Option{totoc}{\default{true}|false}\Default{false}
    This option adds the questions and solutions with their names and numbers
    to the table of contents.
  %% questions-totoc
  \Option{questions-totoc}{\default{true}|false}\Default{false}
    This option adds the questions with their names and numbers to the table
    of contents.
  %% solutions-totoc
  \Option{solutions-totoc}{\default{true}|false}\Default{false}
    This option adds the solutions with their names and numbers to the table
    of contents.
  %% toc-level
  \Option{toc-level}{<toc level>}\Default{subsection}
    This option sets the level in which questions and solutions should appear
    in the table of contents.
  %% questions-toc-level
  \Option{questions-toc-level}{<toc level>}\Default{subsection}
    This option sets the level in which questions should appear in the table
    of contents.
  %% solutions-toc-level
  \Option{solutions-toc-level}{<toc level>}\Default{subsection}
    This option sets the level in which solutions should appear in the table
    of contents.
  %% use-ref
  \Option{use-ref}{\default{true}|false}\Default{false}
    enable referencing to sections and chapters in a way that the references
    can be used with \cmd{printsolutions}, see
    section~\ref{sssec:print_specific_section} for details.
\end{beschreibung}
The \code{toc} options are demonstrated with section~\ref{sec:solutions:list}
and the solutions printed there being listed in the table of contents.

\secidx*{Package Options}

\section{Create Questions/Exercises and their Solutions}
Now, let's start with the most important part: the questions and (possibly)
their respective solutions.
\subsection{The \code{question} Environment}\label{ssec:questions}
Questions are written inside the \code{question} environment:
\begin{beschreibung}
  \Umg{question}[<options>]{\unskip\ma{<points>} ... }
\end{beschreibung}
\begin{beispiel}
 \begin{question}
  This is our very first very difficult to solve question!
 \end{question}
\end{beispiel}
As you can see a heading is automatically created and the question is
numbered.  You can of course change both the numbering and the naming, but
more on that later.

The \code{question} environment takes an optional argument \ma{<points>} that
can be used to assign points to the question (as is common in written exams):
\begin{beispiel}
 \begin{question}{3}
  This is our first difficult question that is worth 3 points!
 \end{question}
\end{beispiel}
These points are saved internally (see section~\ref{sec:points} for reasons
why) and are written to the right margin next to the question heading in the
default setting.

You can also assign bonus points by inserting \code{<point>+<bonus points>} as
argument.
\begin{beispiel}
 \begin{question}{1+1}
  This question is worth 1 point and 1 bonus point.
 \end{question}
 \begin{question}{+3}
  This question is a bonus question. It is worth 3 bonus points.
 \end{question}
\end{beispiel}

\sinceversion{0.3}On additional thing: you might want to define custom
commands that should behave differently if they're inside or outside of the
\code{question} environment.  In this case you can use these commands:
\begin{beschreibung}
  \expandable\Befehl{IfInsideQuestionTF}{<true code>}\ma{<false code>}
  \expandable\Befehl{IfInsideQuestionT}{<true code>}
  \expandable\Befehl{IfInsideQuestionF}{<false code>}
\end{beschreibung}

\subsection{Options to the \code{question} Environment}\secidx[options]{questions}
The \code{question} environment takes one or more of the following options:
\begin{beschreibung}
  \Option[question]{type}{exam|exercise}\Default{exercise}
    determines the type of question and changes the default name of a question
    from ``Exercise'' to ``Question''.  These default names are language
    dependent.
    
    If you use \cmd*{usepackage}[ngerman]{babel}, for example, then the names
    are ``\"Ubung and ``Aufgabe''.
  \Option[question]{name}{<name>}\Default{}
    sets a custom name.  All predefined names are discarded.
  \Option[question]{print}{\default{true}|false}\Default{true}
    prints or hides the question.
  \Option[question]{ID}{<id>}\Default{}
    assigns a custom \acs{id} to the question.  See section~\ref{ssec:ids} for
    further information.
  \Option[question]{label}{<label>}\Default{}
    Places a \cmd*{label}{<label>} for the question.  This will overwrite any
    label that is placed by the \key{auto-label} option.
  \Option[question]{class}{<class>}\Default{}
    assigns a class to the question.  See section~\ref{sec:classes} for
    further information.
  \Option[question]{topic}{<topic>}\Default{}
    assigns a topic to the question.  See section~\ref{sec:topics} for further
    information.
  \Option[question]{use}{\default{true}|false}\Default{true}
    discards the question.  Or not.
\end{beschreibung}
\begin{beispiel}
 \begin{question}[type=exam]
  This question has the type \texttt{exam}. The default name has changed from
  ``Exercise'' to ``Question''.
 \end{question}
 \begin{question}[name=Fancy name]
  This question has a custom name.
 \end{question}
 \begin{question}[print=false]
  This question is not printed.
 \end{question}
\end{beispiel}

The difference between \key{print} and \key{use} lies behind the scenes: with
\key{print}{false} the question is not printed, but it still gets an
individual \ac{id}, is numbered, and a possible solution is saved.  This is
for example useful when you want to print a sample solution for an exam.  With
\key{use}{false} it is fully discarded which means it is not accessible
through an \acs{id} and a possible solution will not be saved.

\secidx*{questions}

\subsection{The \code{solution} Environment}\secidx{solutions}
If you want to save/print (more on the exact usage in
section~\ref{sec:solutions}) a solution you have to use the \code{solution}
environment \emph{after} the question it belongs to and \emph{before} the next
question.
\begin{beschreibung}
  \Umg{solution}[<options>]{...}
\end{beschreibung}
\begin{beispiel}
 \begin{question}[ID=first]\label{qu:question_with_solution}
  This is our first question that gets a solution!
 \end{question}
 \begin{solution}
  This is the solution to exercise~\ref{qu:question_with_solution}!
 \end{solution}
\end{beispiel}
You can see that in the default settings the solution is \emph{not} written to
the document.  It has been saved, though, for possible later usage.  We will
see the solution later!

\subsection{Options to the \code{solution} Environment}\secidx[options]{solutions}
The \code{solutions} environment also has options, namely these:
\begin{beschreibung}
  \Option[solution]{name}{<name>}\Default{}
    sets a custom name.
  \Option[solution]{print}{\default{true}|false}\Default{false}
    prints or hides the solution.
\end{beschreibung}
Their meaning is the same as those for the \code{question} environment.
\begin{beispiel}
 \begin{question}{5}
  The solution to this questions gets printed where it is.
 \end{question}
 \begin{solution}[print]
  See? This solution gets printed where you have put it in the code of
  your document.
 \end{solution}
 \begin{question}{2.5}
  The solution to this questions gets printed where it is \emph{and}
  has a fancy name. Have you noticed that you can assign partial
  points?
 \end{question}
 \begin{solution}[print,name=Fancy name]
  See? This solution gets printed where you have put it and has a fancy
  name!
 \end{solution}
\end{beispiel}
\secidx*{solutions}

\subsection{Setting the Counter}\label{ssec:counter}\secidx{Setting the Counter}
The package option \key{counter-format} allows you to specify how the question
counter is formatted.

The input is an arbitrary string which means you can have anything as counter
number.  However, the letter combinations \code{ch}, \code{se}, \code{qu} and
\code{tsk} are replaced with the counters for the chapter, section, question
or tasks (see part~\ref{part:tasks}), respectively.  While the last one is not
really useful in this case the others allow for a combined numbering.  Each of
these letter combinations can have an optional argument that specifies the
format of the respective counter. \code{1}: \cmd{arabic}, \code{a}:
\cmd{alph}, \code{A}: \cmd{Alph}, \code{r}: \cmd{roman} and \code{R}:
\cmd{Roman}.
\begin{beispiel}
 \SetupExSheets{counter-format=Nr~se~(qu[a])}
 \begin{question}
  A question with a differently formatted number.
 \end{question}
\end{beispiel}
Since the strings associated with the counters are replaced one has to hide
them if they are actually wanted in the counter format.  The easiest way would
to hide them in braces.
\begin{beispiel}
 \SetupExSheets{counter-format={section}\,se~{question}\,(qu[a])}
 \begin{question}
  A question with a yet differently formatted number.
 \end{question}
\end{beispiel}
\secidx*{Setting the Counter}

\subsection{Language Settings}\secidx{Language Settings}
The names of the questions and solutions are language dependent.  If you use
\paket{babel} or \paket{polyglossia} \ExSheets will adapt to the document
language.  \ExSheets has a number of translations but surely not all!  If you
miss a language please drop me a line in an
email\footnote{\href{mailto:contact@mychemistry.eu}{contact@mychemistry.eu}}
containing the \paket{babel} language name and the correct translations for
questions (possibly distinguishing between exercises and exam questions) and
solutions.

Until I implement it you can add something like this to your preamble (example
for Danish) and try if it works:
\begin{beispiel}[code only]
 \DeclareTranslation{Danish}{exsheets-exercise-name}
   {\O{}velse}
 \DeclareTranslation{Danish}{exsheets-question-name}
   {Opgave}
 \DeclareTranslation{Danish}{exsheets-solution-name}
   {Opl\o{}sning}
\end{beispiel}
If this isn't working it means that the language you're using is unknown to
the \paket*{translations} package.  In this case please notify me, too.  You
then can still use the \key{name} options.

\secidx*{Language Settings}

\section{Counting Points}\label{sec:points}\secidx{Counting Points}
\subsection{The Commands}
You have seen in section~\ref{ssec:questions} that you can assign points to a
question.  If you do so these points are printed into the
margin\footnote{Well, not necessarily.  It depends on the heading style you
  have chosen.} and are counted internally.  But there are additional commands
to assign points or bonus points and a number of commands to retrieve the sum
of points and/or bonus points.
\begin{beschreibung}
  \Befehl{addpoints}*{<num>}\newline
    This command can be used to add points assigned to subquestions.
    \cmd{addpoints} will print the points (with ``unit'') \emph{and} add them
    to the sum of all points, \cmd{addpoints}* will only add them but print
    nothing.
  \Befehl{points}*{<num>}\newline
    This command will only print the points (with ``unit'') but won't add them
    to the sum of points.
  \Befehl{addbonus}*{<num>}\newline
    This command can be used to add bonus points assigned to subquestions.
    \cmd{addbonus} will print the points (with ``unit'') \emph{and} add them
    to the sum of all bonus points, \cmd{addbonus}* will only add them but
    print nothing.
  \Befehl{bonus}*{<num>}\newline
    This command will only print the bonus points (with ``unit'') but won't
    add them to the sum of bonus points.
  \Befehl{pointssum}*\newline
    Prints the sum of all points with or without (starred version) ``unit'':
    \pointssum
  \Befehl{currentpointssum}*\newline
    Prints the current sum of points with or without (starred version)
    ``unit'': \currentpointssum
  \Befehl{bonussum}*\newline
    Prints the sum of all bonus points with or without (starred version)
    ``unit'': \bonussum
  \Befehl{currentbonussum}*\newline
    Prints the current sum of bonus points with or without (starred version)
    ``unit'': \currentbonussum
  \Befehl{totalpoints}*\newline
    prints the sum of the points \emph{and} the sum of the bonus points with
    ``unit'': \totalpoints\space The starred version prints the sum of the
    points without ``unit'': \totalpoints*.
\end{beschreibung}
The commands \cmd{pointssum}, \cmd{bonussum} and \cmd{totalpoints} need at
least \emph{two} \LaTeX\ runs to get the sum right.

Suppose you have an exercise worth \points{4} which consists of four questions
listed with an \code{enumerate} environment that are all worth \points{1}
each.  You have two possibilities to display and count them:
\begin{beispiel}
 % uses package `enumitem'
 \begin{question}{4}
  \begin{enumerate}[label=\alph*)]
   \item blah (\points{1})
   \item blah (\points{1})
   \item blah (\points{1})
   \item blah (\points{1})
  \end{enumerate}
 \end{question}
 \begin{question}
  \begin{enumerate}[label=\alph*)]
   \item blah (\addpoints{1})
   \item blah (\addpoints{1})
   \item blah (\addpoints{1})
   \item blah (\addpoints{1})
  \end{enumerate}
 \end{question}
\end{beispiel}

\subsection{Options}\secidx[options]{Counting Points}
\begin{beschreibung}
  \Option[points]{name}{<name>}\Default{P.}
    Choose the ``unit'' for the points.  If you like to differentiate between
    a single point and more than one point you can give a plural ending
    separated with a slash: \key{name}{point/s}.  This sets also the name of
    the bonus points.
  \Option[points]{name-plural}{<plural form of name>}\Default{}
    Instead of forming the plural form with an ending to the singular form
    this option allows to set an extra word for it.  This sets also the plural
    form for the bonus points.
  \Option[points]{bonus-name}{<name>}\Default{P.}
    Choose the ``unit'' for the bonus points.  If you like to differentiate
    between a single point and more than one point you can give a plural
    ending separated with a slash: \key{bonus-name}{point/s}.
  \Option[points]{bonus-plural}{<plural form of name>}\Default{}
    Instead of forming the plural form with an ending to the singular form
    this option allows to set an extra word for it.
  \Option[points]{use-name}{\default{true}|false}\Default{true}
    Don't display the name at all.  Or do.
  \Option[points]{format}{<code>}\Default{\cmd*{@firtsofone}}
    \sinceversion{0.9d}Format number plus name as a whole.  Ideally
    \code{<code>} would end with a command that takes an argument.  Else
    number plus name will be braced.
  \Option[points]{number-format}{<any code>}\Default{}
    This option allows formatting of the number, \eg italics:
    \key{number-format}{\cmd{textit}}.
  \Option[points]{bonus-format}{<any code>}\Default{}
    This option allows formatting of the number of the bonus points, \eg
    italics: \key{bonus-format}{\cmd{textit}}.
  \Option[points]{parse}{\default{true}|false}\Default{true}
    If set to \code{false} the points are not counted and the
    \cmd{totalpoints}, \cmd{pointssum} and \cmd{bonussum} commands won't know
    their value.
  \Option[points]{separate-bonus}{\default{true}|false}\Default{false}
    This option determines whether points and bonus points each get their own
    unit when they appear together (in the margin or with \cmd{totalpoints}).
  \Option[points]{pre-bonus}{<tokens>}\Default{(+}
    Code to be inserted before the bonus points when they follow normal
    points.
  \Option[points]{post-bonus}{<tokens>}\Default{)}
    Code to be inserted after the bonus points when they follow normal
    points.
\end{beschreibung}
\begin{beispiel}
 \SetupExSheets[points]{name=point/s,number-format=\color{red}}
 \begin{question}{1}
  This one's easy so only 1 point can be earned.
 \end{question}
 \begin{question}{7.5}
  But this one's hard! 7.5 points are in there for you!
 \end{question}
\end{beispiel}
\secidx*{Counting Points}

\section{Printing Solutions}\label{sec:solutions}\secidx[print]{solutions}
You have already seen that you can print solutions where they are using the
\key[solution]{print} option.  But \ExSheets offers you quite more
possibilities.

In the next subsections the usage of the command
\begin{beschreibung}
  \Befehl{printsolutions}[<setting>]
\end{beschreibung}
is discussed.

Before we do that a hint: remember that you can set the option \key{print}
globally:
\begin{beispiel}[code only]
 % in the preamble
 \SetupExSheets{solution/print=true}
\end{beispiel}

Now if you want to typeset some text depending on the option being true or not
you can use the following commands:
\begin{beschreibung}
  \expandable\Befehl{PrintSolutionsTF}{<true code>}\ma{<false code>}
  \expandable\Befehl{PrintSolutionsT}{<true code>}
  \expandable\Befehl{PrintSolutionsF}{<false code>}
\end{beschreibung}
They might come in handy if you want two versions of an exercise sheet, one
with the exercises and one with the solutions, and you want to add different
titles to these versions, for instance.

\subsection{Print all}\secidx[print!all]{solutions}
The first and easiest usage of \cmd{printsolutions} is the following:
\begin{beispiel}[code only]
 \printsolutions
\end{beispiel}
There is nothing more to say, really. It prints all solutions you have
specified except those belonging to a question with option \key{use}{false}.
Yes, there's one more point: \cmd{printsolutions} only knows the solutions
that have been set \emph{before} its usage!  This is also true for every usage
explained in the next sections.

\begin{beispiel}[below]
 \printsolutions
\end{beispiel}

\subsection{Print per chapter/section}\secidx[print!per section/chapter]{solutions}
\minisec{Current chapter/section}
If you are not creating an exercise sheet or an exam but are writing a
textbook you maybe want a section at the end of each chapter showing the
solution to the exercises presented in that chapter.  In this case use the
command as follows:
\begin{beispiel}[code only]
 \printsolutions[section]
 % or
 \printsolutions[chapter]
\end{beispiel}
Again, this is pretty much self-explaining.  The solutions to the questions of
the current chapter\footnote{Only if the document class you're using
  \emph{has} chapters, of course!} or section are printed.
\begin{beispiel}
 \begin{question}
  This is the first and only question in this section.
 \end{question}
 \begin{solution}
  This will be one of a few solutions printed by the following call of
  \cmd{printsolutions}.
 \end{solution}
 And now:
 \printsolutions[section]
\end{beispiel}

\minisec{Specific chapter/section}\label{sssec:print_specific_section}
You can also print only the solutions from chapters or sections other than the
current ones.  The syntax is fairly easy:
\begin{beispiel}[below]
 \printsolutions[section={1-7,10}]
 % the same for chapters:
 % \printsolutions[chapter={1-7,10}]
\end{beispiel}
Don't forget that \cmd{printsolutions} cannot know the solutions from
section~10 yet.  It is just used to demonstrate the syntax. You can also use
an open range, \eg something like
\begin{beispiel}[code only]
 \printsolutions[section={-4,10-}]
\end{beispiel}
This would print the solutions from sections~1--4 and from all sections with
number 10\footnote{Or rather where \cmd{value}{section} is 10 or greater --
  the actual counter formatting is irrelevant.} and greater.

There is an obvious disadvantage: you have to know the section numbers!  But
there is a solution: use the package option \key{use-ref}{true}.  Then you can
do something like
\begin{beispiel}[code only]
 % in the preamble:
 \usepackage[use-ref]{exsheets}
 % somewhere in your code after \section{A really cool section title}:
 \label{sec:ReallyCool}
 % somewhere later in your code:
 \printsolutions[section={-\S{sec:ReallyCool}}]
 % which will print all solutions from questions up to and
 % including the really cool section
\end{beispiel}
With the package option \key{use-ref}{true} each usage of \cmd{label} will
create additional labels (one preceded with \code{exse:} and another one with
\code{exch:}) which store the section number and the chapter number,
respectively.  These are used internally by two commands \cmd{S} and \cmd{C}
which refer to the section number and the chapter number the label was created
in.  \emph{These commands are only available as arguments of}
\cmd{printsolutions}.

Since some packages like the well known \paket{hyperref} for example redefine
\cmd{label} \key{use-ref} won't work in together with it.  In this case don't
use \key{use-ref} and set \cmd{exlabel}{<label>} instead to remember the
section/the chapter number.  Its usage is just like \cmd{label}.  So the
safest way is as follows:
\begin{beispiel}[code only]
 % in the preamble:
 \usepackage{exsheets}
 % somewhere in your code after \section{A really cool section title}:
 \exlabel{sec:ReallyCool}
 % somewhere later in your code:
 \printsolutions[section={-\S{sec:ReallyCool}}]
 % which will print all solutions from questions up to and
 % including the really cool section
\end{beispiel}
Please be aware that the labels must be processed in a previous \LaTeX\ run
before \cmd{S} and \cmd{C} can pass them on to \cmd{printsolutions}.

\subsection{Print by \acs{id}}\label{ssec:ids}\secidx[print!by ID]{solutions}
Now comes the best part: you can also print selected solutions!  Every
question has an \acs{id}.  To see which \acs{id} a question has you can call
the following command:
\begin{beschreibung}
  \Befehl{DebugExSheets}{true|false}
  \Befehl{CurrentQuestionID}\newline\sinceversion{0.4a}%
    expands to the current question \acs{id} (after two expansions).
\end{beschreibung}
Let's create some more questions and take a look what this command does:
\begin{beispiel}
 \DebugExSheets{true}
 \begin{question}[ID=nice!]
  A question with a nice \acs{id}!
 \end{question}
 \begin{solution}
  The solution to the question with the nice \acs{id}.
 \end{solution}
 \begin{question}{3.75}
  Yet another question. But this time with quarter points!
 \end{question}
 \begin{solution}
  Yet another solution.
 \end{solution}
\end{beispiel}

So now we can call some specific solutions:
\begin{beispiel}
 \printsolutions[byID={first,nice!,10,14}]
\end{beispiel}
This makes use of the \paket{l3sort} package which at the time of writing is
still considered experimental.  In case you wonder where solution~14 is:
question~14 has no solution given.

If you don't want that the solutions are sorted automatically but appear in
the order given you can use the option
\begin{beschreibung}
  \Option[solution]{sorted}{\default{true}|false}\Default{true}
    Sort solutions given by \acs{id} or don't.
\end{beschreibung}

\section{Dividing Questions into Classes}\label{sec:dividing-concepts}
\subsection{Using Classes}\label{sec:classes}\secidx{Classes}
\secidx[Classes]{Dividing Concepts}
For creating different variants of a written exam or different difficulty
levels of an exercise sheet it comes in handy if one can assign certain
classes to questions and then tell \ExSheets only to use one ore more specific
classes.
\begin{beschreibung}
  \Option{use-classes}{<list of classes>}\Default{}
    When this option is used only the questions belonging to the specified
    classes are printed and have their solutions saved.
\end{beschreibung}
\begin{beispiel}
 \SetupExSheets{use-classes={A,C}}
 \begin{question}[class=A]
  Belonging to class A.
 \end{question}
 \begin{question}[class=B]
  Belonging to class B.
 \end{question}
 \begin{question}[class=C]
  Belonging to class C!
 \end{question}
\end{beispiel}
Questions of classes that are not used are fully discarded. \emph{This also
  means that questions that don't have a class assigned are discarded.}

\secidx*{Classes}
\ExplSyntaxOn
 \bool_set_false:N \g__exsheets_use_classes_bool
\ExplSyntaxOff

\subsection{Using Topics}\label{sec:topics}\secidx{Topics}
\secidx[Topics]{Dividing Concepts}
Similarly to classes one can assign topics to questions. The usage is practically
identical, the semantic meaning is different.
\begin{beschreibung}
  \Option{use-topics}{<list of topics>}\Default{}
    When this option is used only the questions belonging to the specified
    topics are printed and have their solutions saved.
\end{beschreibung}
\begin{beispiel}
 \SetupExSheets{use-topics={trigonometry}}
 \begin{question}[topic=trigonometry]
  A trigonometry question.
 \end{question}
 \begin{question}[topic=arithmetics]
  A arithmetics question
 \end{question}
\end{beispiel}
Questions of topics that are not used are fully discarded. \emph{This also
  means that questions that don't have a topic assigned are discarded.}

If you set both \key{use-classes} and \key{use-topics} then only questions
will be used that \emph{match both categories}.

Ideally one could assign more than one topic to a question but this is
\emph{not} supported yet.
\secidx*{Topics}
\ExplSyntaxOn
 \bool_set_false:N \g__exsheets_use_topics_bool
\ExplSyntaxOff

\subsection{Own Dividing Concepts}\secidx{Dividing Concepts}
\noindent\sinceversion{0.8}Actually both classes and topics are introduced
into \ExSheets internally this way:
\begin{beispiel}[code only]
 \DeclareQuestionClass{class}{classes}
 \DeclareQuestionClass{topic}{topics}
\end{beispiel}
which means you can do the same introducing your own dividing concepts.
\begin{beschreibung}
  \Befehl{DeclareQuestionClass}{<singular name>}\ma{<plural name>}\newline
    Introduces a new dividing concept and defines both new options for the
    \code{question} environment and new global options.
\end{beschreibung}

For example you could decide you want to group your questions according to
their difficulty.  You could place the following line in your preamble:
\begin{beispiel}[code only]
 \DeclareQuestionClass{difficulty}{difficulties}
\end{beispiel}
This would define an option \key{use-difficulties} analogous to
\key{use-classes} and \key{use-topics}.  It would also define an option
\key{difficulty} for the \code{question} environment.  This means you could
now do something like the following:
\begin{beispiel}
 \SetupExSheets{use-difficulties={easy,hard}}
 \begin{question}[difficulty=easy]
  An easy question.
 \end{question}
 \begin{question}[difficulty=medium]
  This one's a bit harder.
 \end{question}
 \begin{question}[difficulty=hard]
  Now let's see if you can solve this one.
 \end{question}
\end{beispiel}
\secidx*{Dividing Concepts}
\ExplSyntaxOn
 \bool_set_false:N \g__exsheets_use_difficulties_bool
\ExplSyntaxOff

\section{Adding and Using Additional Information to Questions}\label{sec:additional_info}
\secidx{Additional Information to Questions}
For managing lots of questions and corresponding solutions it can be very
useful to be able to save and recover additional information to the questions.
This is possible with the following commands.  First the ones for saving:
\begin{beschreibung}
  \Befehl{DeclareQuestionProperty}{<name>}\newline
    This command defines a question property \code{<name>}.  It can only be
    used in the document preamble.
  \Befehl{SetQuestionProperties}{<name>=<value>,...}\newline
    Set the properties for a specific question. this command can only be used
    inside the \code{question} environment.
\end{beschreibung}
Now the commands for recovering the properties:
\begin{beschreibung}
  \Befehl{QuestionNumber}{<id>}\newline
    Recover the number of the question with the \acs{id} \code{<id>}.  The
    number is displayed according to the format set with
    \key{counter-format}.
  \Befehl{GetQuestionProperty}{<name>}\ma{<id>}\newline
    Recover the property \code{<name>} of the question with the \acs{id}
    \code{<id>}.  Of course the property must have been declared before.
\end{beschreibung}

Let's say we have declared the properties \code{notes}, \code{reference} and
\code{topic}.  By default the property \code{points} is available and gets the
value of the optional argument of the \code{question} environment.

We can now do the following:
\begin{beispiel}
 % uses `biblatex'
 \begin{question}[ID=center,topic=LaTeX]{3}
  Explain how you could center text in a \LaTeX\ document.
  \SetQuestionProperties{
    topic     = \TeX/\LaTeX ,
    notes     = {How to center text.},
    reference = {\textcite{companion}}}
 \end{question}
 \begin{solution}
  To center a short part of the text body one can use the \texttt{center}
  environment (\points{1}). Inside an environment like \texttt{table} one
  should use \texttt{\string\centering} (\points{1}). For single lines
  there is also the \texttt{\string\centerline} command (\points{1}).
 \end{solution}
 \begin{question}[ID=knuthbooks,topic=LaTeX]{2}
  Name two books by D.\,E.\,Knuth.
  \SetQuestionProperties{
    topic     = \TeX/\LaTeX ,
    notes     = {Books by Knuth.},
    reference = {\textcite{knuth:ct:a,knuth:ct:b,knuth:ct:c,knuth:ct:d,knuth:ct:e}}}
 \end{question}
 \begin{solution}
  For example two volumes from \citetitle{knuth:ct}:
  \citetitle{knuth:ct:a,knuth:ct:b,knuth:ct:c,knuth:ct:d,knuth:ct:e}. Each valid
  answer is worth \points{1}
 \end{solution}
\end{beispiel}

It is now possible to recover these values later:
\begin{beispiel}
 % uses `booktabs'
 \begin{center}
  \begin{tabular}{lll}
   \toprule
    Question & Property & \\
   \midrule
    \QuestionNumber{center}
      & Points     & \GetQuestionProperty{points}{center} \\
      & Topic      & \GetQuestionProperty{topic}{center} \\
      & References & \GetQuestionProperty{reference}{center} \\
      & Note       & \GetQuestionProperty{notes}{center} \\
   \midrule
    \QuestionNumber{knuthbooks}
      & Points     & \GetQuestionProperty{points}{knuthbooks} \\
      & Topic      & \GetQuestionProperty{topic}{knuthbooks} \\
      & References & \GetQuestionProperty{reference}{knuthbooks} \\
      & Note       & \GetQuestionProperty{notes}{knuthbooks} \\
   \bottomrule
  \end{tabular}
 \end{center}
\end{beispiel}

\sinceversion{0.7a}If you use the package option \key{auto-label} the
properties \code{ref} and \code{pageref} are predefined which will call the
corresponding \cmd*{ref} or \cmd*{pageref}, respectively.

Please note that these properties \emph{are not the same} as the dividing
concepts explained in section~\ref{sec:dividing-concepts} although they may
seem similar in meaning or even have the same name.

\sinceversion{0.3}There are additional commands that might prove useful:
\begin{beschreibung}
  \Befehl{ForEachQuestion}{<code to be executed for each used question>}\newline
    Inside the argument one can refer to the \ac{id} of a question with
    \code{\#1}.  Beware that this command only knows of questions used before
    it is issued.
  \expandable\Befehl{numberofquestions}\newline
    Returns the current number of used questions.  Beware that this command
    only knows of questions used before it is issued.
  \expandable\Befehl{iflastquestion}{<true code>}\ma{<false code>}\newline
    Although this command is available in the whole document it is only useful
    inside \cmd{ForEachQuestion}.  It tells you if the end of the loop is
    reached or not.
\end{beschreibung}
For example one could use these commands to create a grading table:
\begin{beispiel}[code only]
 \begin{tabular}{|l|*{\numberofquestions}{c|}c|}\hline
  Question &
    \ForEachQuestion{\QuestionNumber{#1}\iflastquestion{}{&}} &
    Total \\ \hline
  Points   &
    \ForEachQuestion{\GetQuestionProperty{points}{#1}\iflastquestion{}{&}} &
    \pointssum* \\ \hline
  Reached  &
    \ForEachQuestion{\iflastquestion{}{&}} & \\ \hline
 \end{tabular}
\end{beispiel}
For four questions the table now would look similar to
figure~\ref{fig:grading-table}.

\begin{figure}[ht]
 \centering
 \begin{tabular}{|l|*{4}{c|}c|}\hline
  Question & 1. & 2. & 3. & 4. & Total \\ \hline
  Points   &  3 &  5 & 10 &  8 & 26 \\ \hline
  Reached  &    &    &    &    &    \\ \hline
 \end{tabular}
 \caption{An example for a grading table. (Actually this is a fake. See the
   \code{grading-table.tex} file shipped with exsheets for the real use case.)}
 \label{fig:grading-table}
\end{figure}
\secidx*{Additional Information to Questions}

\section{Variations of an Exam}\secidx{Variations}
\noindent\sinceversion{0.6}It is a quite common task to design an exam in two
different variants.  This is of course possible with \ExSheets' classes (see
section~\ref{sec:classes}).  However, often not the whole question is to be
different but only small details, the numbers in a maths exam, say.  For this
purpose \ExSheets provides the following commands:
\begin{beschreibung}
  \Befehl{SetVariations}{<num>}\newline
    Set the number of different variants.  This will determine how many
    arguments the command \cmd*{vary} will get.  \code{<num>} must at least be
    \code{2} and is initially set to \code{2}.
  \Befehl{variant}{<num>}\newline
    Choose the active variant.  The argument must be a number between \code{1}
    and the number set with \cmd*{SetVariations}.  Initially set to \code{1}.
  \Befehl{vary}{<variant 1>}\ma{<variant 2>}\newline
    This command is the one actually used in the document.  It has a number of
    required arguments equal to the number set with \cmd*{SetVariations}.  All
    of its arguments are discarded except the one specified with
    \cmd*{variant}.
  \Befehl{lastvariant}\newline% TODO: richtige version einfügen!
    \sinceversion{0.7b}Each time \cmd{vary} is called it stores the value it
    chose in \cmd{lastversion}.  This might be convenient to use if one
    otherwise would have to repeatedly write the same \cmd{vary}.
\end{beschreibung}

\begin{beispiel}
 \SetVariations{6}%
 \variant{6}\vary{A}{B}{C}{D}{E}{F}
 (last variant: \lastvariant)
 \variant{1}\vary{A}{B}{C}{D}{E}{F}
 (last variant: \lastvariant)
 \variant{5}\vary{A}{B}{C}{D}{E}{F}
 (last variant: \lastvariant)
 \variant{2}\vary{A}{B}{C}{D}{E}{F}
 (last variant: \lastvariant)
 \variant{4}\vary{A}{B}{C}{D}{E}{F}
 (last variant: \lastvariant)
 \variant{3}\vary{A}{B}{C}{D}{E}{F}
 (last variant: \lastvariant)
\end{beispiel}
\secidx*{Variations}

\section{A Grade Distribution}\secidx{Grade Distribution}
Probably this is a rather esoteric feature but it could proof useful in some
cases.  Suppose you are a German math teacher and want to grade exactly
corresponding to the number of points relative to the sum of total points,
regardless of how big that might be.  You could do something like this to
present your grading decisions for the exam:
\begin{beispiel}
 % preamble:
 % \DeclareRelGrades{
 %   1     = 1 ,
 %   {1,5} = .9167 ,
 %   2     = .8333 ,
 %   {2,5} = .75 ,
 %   3     = .6667 ,
 %   {3,5} = .5833 ,
 %   4     = .5
 % }
 \small\setlength\tabcolsep{2pt}
 \begin{tabular}{r|*8c}
  Punkte
    & $\grade*{1}$      & $\le\grade*{1}$ & $\le\grade*{1,5}$ & $\le\grade*{2}$
    & $\le\grade*{2,5}$ & $\le\grade*{3}$ & $\le\grade*{3,5}$ & $<\grade*{4}$ \\
  Note
    & 1 & 1--2 & 2 & 2--3 & 3 & 3--4 & 4 & 5
 \end{tabular}
\end{beispiel}

These are the available commands and options:
\begin{beschreibung}
  \Befehl{DeclareRelGrades}{<grade>=<num>,...}\newline
    This command is used to define grades and assign the percentage of total
    points to them.
  \Befehl{grade}*{<grade>}\newline
    Gives the number of points corresponding to a grade depending on the value
    of \cmd{pointssum} with or without (starred version) ``unit''.
  \Option[grades]{round}{<num>}\Default{0}
    The number of decimals the points of a grade are rounded to.  This doesn't
    apply to the maximum number of points if the rounded number would be
    bigger than the actual sum.
  \Option[grades]{half}{\default{true}|false}\Default{false}
    If set to \code{true} points are rounded either to full or to half
    points.
\end{beschreibung}
\secidx*{Grade Distribution}

\section{Selectively Include Questions from External Files}\label{sec:include}
\secidx{Include from External Files}
\subsection{Caveat}
I need to say some words of caution: the \cmd{includequestions} that will be
presented shortly is probably \ExSheets' most experimental one at the time of
writing (\today).  Thanks to feedback of users it is constantly improved and
bugs are fixed.  It is not a very efficient way to insert question regarding
performance and you shouldn't wonder if compilation slows down when you use
it.  It probably needs to be re-written all over but on the one hand that
would introduce new bugs and on the other hand for the time being I don't have
the capacities, anyway, so you'll have to live it, I'm afraid.

\subsection{How it works}
Suppose you have one or more files with questions prepared to use them as a
kind of database.  One for class A, say, one for class B, one for class C and
so one, something like this:
\begin{beispiel}[code only]
 % this is file classA.tex
 \begin{question}[class=A,ID=A1,topic=X]
  First question of class A, topic X.
 \end{question}
 \begin{solution}
  First solution of class A.
 \end{solution}
 \begin{question}[class=A,ID=A2,topic=Y]
  Second question of class A, topic Y.
 \end{question}
 \begin{solution}
  Second solution of class A.
 \end{solution}
 ...
 % end of file classA.tex
 \endinput
\end{beispiel}
You can of course just \cmd{input} or \cmd{include} it but that would of
course include the whole file into your document.  But would't it be nice to
just include selected questions?  Or maybe a five random questions from the
file?  That is possible with the following command:
\begin{beschreibung}
  \Befehl{includequestions}[<options>]{<list of filenames>}
\end{beschreibung}
If you use it without options it will have the same effect as
\cmd{input}.  There are however the following options:
\begin{beschreibung}
  \Option[include]{all}{\default{true}|false}
  \Option[include]{IDs}{<list of IDs>}\Default{}
    Includes only the specified questions.
  \Option[include]{random}{<num>}\Default{}
    Includes \code{<num>} randomly selected questions.  This option uses the
    \paket{pgfcore} package to create the pseudo-random numbers.
  \Option[include]{exclude}{<list of IDs>}\Default{}
    Questions who's \acp{id} are specified here are \emph{not} included.  This
    option can be combined with the \key{random} option.
\end{beschreibung}
The usage should be self-explainable:
\begin{beispiel}[code only]
 % include questions A1, A3 and A4:
 \includequestions[IDs={A1,A3,A4}]{classA.tex}
 % or include 3 random questions:
 \includequestions[random=3]{classA}
\end{beispiel}
In order to be able to select the questions \ExSheets needs to \cmd{input} the
file twice.  The first time the available questions are determined, the second
time the selected questions are used.  This unfortunately means that anything
that is \emph{not} part of a question or solution is also input twice.  Either
don't put anything else into the file or use one of the following commands for
control:
\begin{beschreibung}
 \Befehl{PrintIfIncludeActiveTF}\ma{<true code>}\ma{<false code>}
 \Befehl{PrintIfIncludeActiveT}\ma{<true code>}
 \Befehl{PrintIfIncludeActiveF}\ma{<false code>}
\end{beschreibung}

The selection can be refined further by selecting questions belonging to a
specific class of questions (see section~\ref{sec:dividing-concepts}) before
using \cmd{includequestions}.

\sinceversion{0.8}After you've used \cmd{includequestions} the \acp{id} of the
included questions is available as an unordered comma separated list in the
following macro:
\begin{beschreibung}
  \Befehl{questionsincludedlast}\newline
    Unordered comma separated list of question \acp{id} included with the last
    usage of \cmd{includequestions}.
\end{beschreibung}
\secidx*{Include from External Files}

\section{Own Question/Solution Pairs}
\noindent\changedversion{0.9}\ExSheets provides he possibility to create new
environments that behave like the \code{question} and \code{solution}
environments.  This would allow, for example, to define a
\code{question*}/\code{solution*} environment pair for bonus questions.  The
following commands may be used in the document preamble:
\begin{beschreibung}
  \Befehl{NewQuSolPair}{<question>}\oa{<question options>}\oa{<general options>}%
    \ma{<solution>}\oa{<solution options>}\oa{<general options>}
  \Befehl{RenewQuSolPair}{<question>}\oa{<question options>}\oa{<general options>}%
    \ma{<solution>}\oa{<solution options>}\oa{<general options>}
\end{beschreibung}
The standard environments are defined as follows:
\begin{beispiel}[code only]
 \NewQuSolPair{question}{solution}
\end{beispiel}

Let's say we want the possibility to add bonus questions.  A simple way would
be to define starred variants that add a star in the margin left to the title:
\begin{beispiel}
 % preamble:
 % \NewQuSolPair
 %   {question*}[name=\protect\llap{$\bigstar$\space}Bonus Question]
 %   {solution*}[name=\protect\llap{$\bigstar$\space}Solution]
 \begin{question*}
  This is a bonus question.
 \end{question*}
 \begin{solution*}[print]
  This is what the solution looks like.
 \end{solution*}
\end{beispiel}
As you can see the environments take the same options as are described for the
standard \code{question} and \code{solution} environments.

\section{Filling in the Blanks}\secidx{Blanks}
\subsection{Cloze}\secidx[Cloze]{Blanks}
\noindent\changedversion{0.4}Both in exercise sheets and in exams it is
sometimes desirable to be able to create \blank{blanks} that have to be filled
in.  Or maybe some more lines: \blank[width=5\linewidth]{}

\begin{beschreibung}
  \Befehl{blank}*[<options>]{<text to be filled in>}\newline
    creates a blank in normal text or in a question but fills the text of its
    argument if inside a solution.  If used at the \emph{begin of a paragraph}
    \cmd{blank} will do two things: it will set the linespread according to an
    option explained below and will insert \cmd*{par} after the lines.  If you
    don't want that use the starred version.
\end{beschreibung}

The options are these:
\begin{beschreibung}
  \Option[blank]{style}{line|wave|dline|dotted|dashed}\Default{line}
    The style of the line.  This uses the corresponding command from the
    \paket{ulem} package and is the whole reason why \ExSheets loads it in the
    first place.
  \Option[blank]{scale}{<num>}\Default{1}
    Scales the width of the blank by factor \code{<num>} unless the width is
    explicitly set.
  \Option[blank]{width}{<dim>}\Default{}
    The width of the line.  If it is not used the width of the filled in text
    is used.
  \Option[blank]{linespread}{<num>}\Default{1}
    Set the linespread for the blank lines.  This only has an effect if
    \cmd{blank} is used at the begin of a paragraph.
\end{beschreibung}
\begin{beispiel}
 \begin{question}
  Try to fill in \blank[width=4cm]{these} blanks. All of them
  \blank[style=dotted]{are created} by using the \cmd{blank}
  \blank[style=dashed]{command}.
 \end{question}
 \begin{solution}[print]
  Try to fill in \blank[width=4cm]{these} blanks. All of them
  \blank[style=dotted]{are created} by using the \cmd{blank}
  \blank[style=dashed]{command}.
 \end{solution}
\end{beispiel}
A number of empty lines are easily created by setting the width option:
\begin{beispiel}
 
 \blank[width=4.8\linewidth,linespread=1.5]{}
\end{beispiel}

\subsection{Vertical Space for answers}\secidx[vertical space]{Blanks}
\noindent\sinceversion{0.3}When you're creating an exam you might want to add
some vertical space where the students can write down their answers.  While
you can always use \cmd*{vspace} this is not always handy when the space left
on the page is less than you want.  In this case it would be nice if a) there
would be no warning and b) the rest of the space would be added at the top of
the next page.  This is what the following command is for:
\begin{beschreibung}
  \Befehl{examspace}*{<dim>}\newline
    Add space as specified in \ma{<dim>}. If the space available on the
    current page is not enough the rest of the space will be added at the top
    of the next page.  The starred version will silently drop any leftover
    space instead of adding it to the next page.
\end{beschreibung}
\begin{beispiel}
 \begin{question}
  What do you think of this feature?
  \examspace{3cm}
 \end{question}
 This line comes after the space.
\end{beispiel}
\secidx*{Blanks}

\section{Styling your Exercise/Exam Sheets}\label{part:style}
\subsection{Background}
The \ExSheets package makes extensive use of \LaTeX3's coffins\footnote{See
  the documentation to the \paket*{xcoffins} package for more information on
  that.} as well as its templates concept\footnote{Have a look into the
  documentation to the \paket*{xtemplate} package.}.  The latter allows a
rather easy extension and customization of some of \ExSheets' environments.
To be more precise: you can define your own instances for the headings used
for questions and solutions and for the \code{tasks} environment.

What this package doesn't provide is changing the background of questions or
framing them.  But this is easily possible using the \paket{mdframed} package
and its \cmd*{surroundwithmdframed} command.

\subsection{The \code{exsheets-headings} Object}\label{sec:exsheets-headings}
\secidx{The `exsheets-headings' Object}
\ExSheets defines the object \code{exsheets-headings} and one template for it,
the `default' template.  The package also defines two instances of this
template, the `block' instance and the `runin' instance.

\begin{beispiel}
 \SetupExSheets{headings=block}
 \begin{question}{1}
  a `block' heading
 \end{question}
 \SetupExSheets{headings=runin}
 \begin{question}{1}
  a `runin' heading
 \end{question}
\end{beispiel}

\subsubsection{Available Options}
This section only lists the options that can be used when defining an instance
of the `default' template.  The following subsections will give loads of
examples of their usage.  The options are listed in the definition for the
template interface:

\begin{beispiel}[code only]
 \DeclareTemplateInterface{exsheets-heading}{default}{3}
  {
    % option         : type      = default
    inline           : boolean   = false ,
    runin            : boolean   = false ,
    indent-first     : boolean   = false ,
    toc-reversed     : boolean   = false ,
    vscale           : real      = 1     ,
    above            : length    = 2pt   ,
    below            : length    = 2pt   ,
    main             : tokenlist =       ,
    pre-code         : tokenlist =       ,
    post-code        : tokenlist =       ,
    title-format     : tokenlist =       ,
    title-pre-code   : tokenlist =       ,
    title-post-code  : tokenlist =       ,
    number-format    : tokenlist =       ,
    number-pre-code  : tokenlist =       ,
    number-post-code : tokenlist =       ,
    points-format    : tokenlist =       ,
    points-pre-code  : tokenlist =       ,
    points-post-code : tokenlist =       ,
    join             : tokenlist =       ,
    attach           : tokenlist =
  }
\end{beispiel}


Each heading is built with at most four coffins available with the names
`main', `title', `number' and `points'.  Those coffins place possibly the
whole heading, the title, the question number and the assigned points.  The
only coffin that's always typeset is the `main' coffin, which is empty per
default.

Coffins can be joined (two become one, the first extends its bounding box to
contain the second) using the following syntax:
\begin{beispiel}[code only]
 join = coffin1[handle11,handle12]coffin2[handle21,handle22](x-offset,y-offset)
\end{beispiel}
The syntax for attaching (two become one, the first does \emph{not} extend its
bounding box around the second) is the same.

More on coffin handles is described in the documentation for the
\paket{xcoffins}.  Figure~\ref{fig:handles} briefly demonstrates the available
handle pairs.

\begin{figure}[ht]
 \centering
 \parbox{4.5cm}{%
   \NewCoffin\ExampleCoffin
   \SetHorizontalCoffin\ExampleCoffin{\color{gray!30}\rule{4cm}{4cm}}%
   \DisplayCoffinHandles\ExampleCoffin{blue}%
 }
 \caption{Available handles for a horizontal coffin.}\label{fig:handles}
\end{figure}

The following subsections will show all definitions of the instances available
with the package option \key{load-headings} and how they look.  This will
hopefully give you enough ideas to create your own instance if you want to
have another heading style than the ones available.

Of you use the option \key{load-headings} each of the following instances is
available through the option \key{headings}{<instance>}.

The following examples use a sample text defined as follows:
\begin{beispiel}[code only]
 \def\s{This is some sample text we will use to create a somewhat
   longer text spanning a few lines.}
 \def\sample{\s\ \s\par\s}
\end{beispiel}
\def\s{This is some sample text we will use to create a somewhat longer text
 spanning a few lines.}
\def\sample{\s\ \s\par\s}

\subsubsection{The `block' Instance}
\begin{beispiel}[code only]
 \DeclareInstance{exsheets-heading}{block}{default}
  {
    join             = { title[r,B]number[l,B](1ex,0pt) } ,
    attach           =
      {
        main[l,vc]title[l,vc](0pt,0pt) ;
        main[r,vc]points[l,vc](\marginparsep,0pt)
      }
  }
\end{beispiel}
\SetupExSheets{headings=block}
\begin{question}{1}
 A `block' heading. \sample
\end{question}

\subsubsection{The `runin' Instance}
\begin{beispiel}[code only]
 \DeclareInstance{exsheets-heading}{runin}{default}
  {
    runin            = true ,
    number-post-code = \space ,
    attach           =
      { main[l,vc]points[l,vc](\linewidth+\marginparsep,0pt) } ,
    join             =
      {
        main[r,vc]title[r,vc](0pt,0pt) ;
        main[r,vc]number[l,vc](1ex,0pt)
      }
  }
\end{beispiel}
\SetupExSheets{headings=runin}
\begin{question}{1}
 A `runin' heading. \sample
\end{question}

\subsubsection{The `simple' Instance}
\begin{beispiel}[code only]
 \DeclareInstance{exsheets-heading}{simple}{default}
  {
    title-format     = \normalsize ,
    points-pre-code  = ( ,
    points-post-code = ) ,
    attach           = { main[l,t]number[l,t](0pt,0pt) } ,
    join             =
      {
        number[r,b]title[l,b](1ex,0pt) ;
        main[l,b]points[l,t](1em,0pt)
      }
  }
\end{beispiel}
\SetupExSheets{headings=simple}
\begin{question}{1}
 A `simple' heading. \sample
\end{question}

\subsubsection{The `empty' Instance}
\sinceversion{0.9a}
\begin{beispiel}[code only]
 \DeclareInstance{exsheets-heading}{empty}{default}
  {
    runin  = true ,
    above  = \parskip ,
    below  = \parskip ,
    attach = { main[l,vc]points[l,vc](\linewidth+\marginparsep,0pt) }
  }
\end{beispiel}
\SetupExSheets{headings=empty}
\begin{question}{1}
 An `empty' heading. \sample
\end{question}

\subsubsection{The `block-rev' Instance}
\begin{beispiel}[code only]
 \DeclareInstance{exsheets-heading}{block-rev}{default}
  {
    toc-reversed     = true ,
    join             = { number[r,B]title[l,B](1ex,0pt) } ,
    attach           =
      {
        main[l,vc]number[l,vc](0pt,0pt) ;
        main[r,vc]points[l,vc](\marginparsep,0pt)
      }
  }
\end{beispiel}
\SetupExSheets{headings=block-rev}
\begin{question}{1}
 A `block-rev' heading. \sample
\end{question}

\subsubsection{The `block-wp' Instance}
\begin{beispiel}[code only]
 \DeclareInstance{exsheets-heading}{block-wp}{default}
  {
    points-pre-code  = ( ,
    points-post-code = ) ,
    join             =
      {
        title[r,B]number[l,B](1ex,0pt) ;
        title[r,B]points[l,B](1ex,0pt)
      } ,
    attach           = { main[l,vc]title[l,vc](0pt,0pt) }
  }
\end{beispiel}
\SetupExSheets{headings=block-wp}
\begin{question}{1}
 A `block-wp' heading. \sample
\end{question}

\subsubsection{The `block-wp-rev' Instance}
\begin{beispiel}[code only]
 \DeclareInstance{exsheets-heading}{block-wp-rev}{default}
  {
    toc-reversed     = true ,
    points-pre-code  = ( ,
    points-post-code = ) ,
    join             =
      {
        number[r,B]title[l,B](1ex,0pt) ;
        number[r,B]points[l,B](1ex,0pt)
      } ,
    attach           = { main[l,vc]number[l,vc](0pt,0pt) }
  }
\end{beispiel}
\SetupExSheets{headings=block-wp-rev}
\begin{question}{1}
 A `block-wp-rev' heading. \sample
\end{question}

\subsubsection{The `block-nr' Instance}
\begin{beispiel}[code only]
 \DeclareInstance{exsheets-heading}{block-nr}{default}
  {
    attach           =
      {
        main[l,vc]number[l,vc](0pt,0pt) ;
        main[r,vc]points[l,vc](\marginparsep,0pt)
      }
  }
\end{beispiel}
\SetupExSheets{headings=block-nr}
\begin{question}{1}
 A `block-nr' heading. \sample
\end{question}

\subsubsection{The `block-nr-wp' Instance}
\begin{beispiel}[code only]
 \DeclareInstance{exsheets-heading}{block-nr-wp}{default}
  {
    points-pre-code  = ( ,
    points-post-code = ) ,
    join             = { number[r,vc]points[l,vc](1ex,0pt) } ,
    attach           = { main[l,vc]number[l,vc](0pt,0pt) }
  }
\end{beispiel}
\SetupExSheets{headings=block-nr-wp}
\begin{question}{1}
 A `block-nr-wp' heading. \sample
\end{question}

\subsubsection{The `runin-rev' Instance}
\begin{beispiel}[code only]
 \DeclareInstance{exsheets-heading}{runin-rev}{default}
  {
    toc-reversed     = true ,
    runin            = true ,
    title-post-code  = \space ,
    attach           =
      { main[l,vc]points[l,vc](\linewidth+\marginparsep,0pt) } ,
    join             =
      {
        main[r,vc]number[r,vc](0pt,0pt) ;
        main[r,vc]title[l,vc](1ex,0pt)
      }
  }
\end{beispiel}
\SetupExSheets{headings=runin-rev}
\begin{question}{1}
 A `runin-rev' heading. \sample
\end{question}

\subsubsection{The `runin-wp' Instance}
\begin{beispiel}[code only]
 \DeclareInstance{exsheets-heading}{runin-wp}{default}
  {
    runin            = true ,
    points-pre-code  = ( ,
    points-post-code = )\space ,
    join             =
      {
        main[r,vc]title[r,vc](0pt,0pt) ;
        main[r,vc]number[l,vc](1ex,0pt) ;
        main[r,vc]points[l,vc](1ex,0pt)
      }
  }
\end{beispiel}
\SetupExSheets{headings=runin-wp}
\begin{question}{1}
 A `runin-wp' heading. \sample
\end{question}

\subsubsection{The `runin-wp-rev' Instance}
\begin{beispiel}[code only]
 \DeclareInstance{exsheets-heading}{runin-wp-rev}{default}
  {
    toc-reversed     = true ,
    runin            = true ,
    points-pre-code  = ( ,
    points-post-code = )\space ,
    join             =
      {
        main[r,vc]number[r,vc](0pt,0pt) ;
        main[r,vc]title[l,vc](1ex,0pt) ;
        main[r,vc]points[l,vc](1ex,0pt)
      }
  }
\end{beispiel}
\SetupExSheets{headings=runin-wp-rev}
\begin{question}{1}
 A `runin-wp-rev' heading. \sample
\end{question}

\subsubsection{The `runin-nr' Instance}
\begin{beispiel}[code only]
 \DeclareInstance{exsheets-heading}{runin-nr}{default}
  {
    runin            = true ,
    number-post-code = \space ,
    attach           =
      { main[l,vc]points[l,vc](\linewidth+\marginparsep,0pt) } ,
    join             = { main[r,vc]number[l,vc](0pt,0pt) }
  }
\end{beispiel}
\SetupExSheets{headings=runin-nr}
\begin{question}{1}
 A `runin-nr' heading. \sample
\end{question}

\newpage
\subsubsection{The `runin-fixed-nr' Instance}
\begin{beispiel}[code only]
 \DeclareInstance{exsheets-heading}{runin-fixed-nr}{default}
  {
    runin            = true ,
    number-pre-code  = \hbox to 2em \bgroup ,
    number-post-code = \hfil\egroup ,
    attach           =
      { main[l,vc]points[l,vc](\linewidth+\marginparsep,0pt) } ,
    join             = { main[r,vc]number[l,vc](0pt,0pt) }
  }
\end{beispiel}
\SetupExSheets{headings=runin-fixed-nr}
\begin{question}{1}
 A `runin-fixed-nr' heading. \sample
\end{question}

\subsubsection{The `runin-nr-wp' Instance}
\needspace{2\baselineskip}
\begin{beispiel}[code only]
 \DeclareInstance{exsheets-heading}{runin-nr-wp}{default}
  {
    runin            = true ,
    points-pre-code  = ( ,
    points-post-code = )\space ,
    join             =
      {
        main[r,vc]number[l,vc](0pt,0pt) ;
        main[r,vc]points[l,vc](1ex,0pt)
      }
  }
\end{beispiel}
\SetupExSheets{headings=runin-nr-wp}
\begin{question}{1}
 A `runin-nr-wp' heading. \sample
\end{question}

\subsubsection{The `inline' Instance}
\sinceversion{0.5}
\begin{beispiel}[code only]
 \DeclareInstance{exsheets-heading}{inline}{default}
  {
    inline           = true ,
    number-pre-code  = \space ,
    number-post-code = \space ,
    join             =
      {
        main[r,vc]title[r,vc](0pt,0pt) ;
        main[r,vc]number[l,vc](0pt,0pt)
      }
  }
\end{beispiel}
\SetupExSheets{headings=inline}
Text before
\begin{question}{1}
 An `inline' heading. \sample
\end{question}
 Text after

\subsubsection{The `inline-wp' Instance}
\sinceversion{0.5}
\begin{beispiel}[code only]
 \DeclareInstance{exsheets-heading}{inline-wp}{default}
  {
    inline           = true ,
    number-pre-code  = \space ,
    number-post-code = \space ,
    points-pre-code  = ( ,
    points-post-code = )\space ,
    join             =
      {
        main[r,vc]title[r,vc](0pt,0pt) ;
        main[r,vc]number[l,vc](0pt,0pt) ;
        main[r,vc]points[l,vc](0pt,0pt)
      }
  }
\end{beispiel}
\SetupExSheets{headings=inline-wp}
Text before
\begin{question}{1}
 An `inline-wp' heading. \sample
\end{question}
 Text after

\subsubsection{The `inline-nr' Instance}
\sinceversion{0.5}
\begin{beispiel}[code only]
 \DeclareInstance{exsheets-heading}{inline-nr}{default}
  {
    inline           = true ,
    number-post-code = \space ,
    join             = { main[r,vc]number[l,vc](0pt,0pt) }
  }
\end{beispiel}
\SetupExSheets{headings=inline-nr}
Text before
\begin{question}{1}
 An `inline-nr' heading. \sample
\end{question}
 Text after

\subsubsection{The `centered' Instance}
\begin{beispiel}[code only]
 \DeclareInstance{exsheets-heading}{centered}{default}
  {
    join             = { title[r,B]number[l,B](1ex,0pt) } ,
    attach           =
      {
        main[hc,vc]title[hc,vc](0pt,0pt) ;
        main[r,vc]points[l,vc](\marginparsep,0pt)
      }
  }
\end{beispiel}
\SetupExSheets{headings=centered}
\begin{question}{1}
 A `centered' heading. \sample
\end{question}

\subsubsection{The `centered-wp' Instance}
\begin{beispiel}[code only]
 \DeclareInstance{exsheets-heading}{centered-wp}{default}
  {
    points-pre-code  = ( ,
    points-post-code = ) ,
    join             =
      {
        title[r,B]number[l,B](1ex,0pt) ;
        title[r,B]points[l,B](1ex,0pt)
      } ,
    attach           = { main[hc,vc]title[hc,vc](0pt,0pt) }
  }
\end{beispiel}
\SetupExSheets{headings=centered-wp}
\begin{question}{1}
 A `centered-wp' heading. \sample
\end{question}

\subsubsection{The `margin' Instance}
\begin{beispiel}[code only]
 \DeclareInstance{exsheets-heading}{margin}{default}
  {
    runin            = true ,
    number-post-code = \space ,
    points-pre-code  = ( ,
    points-post-code = )\space ,
    join             = { title[r,b]number[l,b](1ex,0pt) } ,
    attach           =
      {
        main[l,vc]title[r,vc](0pt,0pt) ;
        main[l,b]points[r,t](0pt,0pt)
      }
  }
\end{beispiel}
\SetupExSheets{headings=margin}
\begin{question}{1}
 A `margin' heading. \sample
\end{question}

\subsubsection{The `margin-nr' Instance}
\begin{beispiel}[code only]
 \DeclareInstance{exsheets-heading}{margin-nr}{default}
  {
    runin  = true ,
    attach =
      {
        main[l,vc]number[r,vc](-1ex,0pt) ;
        main[r,vc]points[l,vc](\linewidth+\marginparsep,0pt)
      }
  }
\end{beispiel}
\SetupExSheets{headings=margin-nr}
\begin{question}{1}
 A `margin-nr' heading. \sample
\end{question}

\subsubsection{The `raggedleft' Instance}
\begin{beispiel}[code only]
 \DeclareInstance{exsheets-heading}{raggedleft}{default}
  {
    join             = { title[r,B]number[l,B](1ex,0pt) } ,
    attach           =
      {
        main[r,vc]title[r,vc](0pt,0pt) ;
        main[r,vc]points[l,vc](\marginparsep,0pt)
      }
  }
\end{beispiel}
\SetupExSheets{headings=raggedleft}
\begin{question}{1}
 A `raggedleft' heading. \sample
\end{question}

\subsubsection{The `fancy' Instance}
\begin{beispiel}[code only]
 \DeclareInstance{exsheets-heading}{fancy}{default}
  {
    toc-reversed     = true ,
    indent-first     = true ,
    vscale           = 2 ,
    pre-code         = \rule{\linewidth}{1pt} ,
    post-code        = \rule{\linewidth}{1pt} ,
    title-format     = \large\scshape\color{exsheetsred} ,
    number-format    = \large\bfseries\color{exsheetsblue} ,
    points-format    = \itshape ,
    join             = { number[r,B] title[l,B] (1ex,0pt) } ,
    attach           =
      {
        main[hc,vc]number[hc,vc](0pt,0pt) ;
        main[l,vc]points[r,vc](-\marginparsep,0pt)
      }
  }
\end{beispiel}
\SetupExSheets{headings=fancy}
\begin{question}{1}
 A `fancy' heading. \sample
\end{question}

\subsubsection{The `fancy-wp' Instance}
\begin{beispiel}[code only]
 \DeclareInstance{exsheets-heading}{fancy-wp}{default}
  {
    toc-reversed     = true ,
    indent-first     = true ,
    vscale           = 2 ,
    pre-code         = \rule{\linewidth}{1pt} ,
    post-code        = \rule{\linewidth}{1pt} ,
    title-format     = \large\scshape\color{exsheetsred} ,
    number-format    = \large\bfseries\color{exsheetsblue} ,
    points-format    = \itshape ,
    points-pre-code  = ( ,
    points-post-code = ) ,
    join             =
      {
        number[r,B]title[l,B](1ex,0pt) ;
        number[r,B]points[l,B](1ex,0pt)
      } ,
    attach           = { main[hc,vc]number[hc,vc](0pt,0pt) }
  }
\end{beispiel}
\SetupExSheets{headings=fancy-wp}
\begin{question}{1}
 A `fancy-wp' heading. \sample
\end{question}
\secidx*{The `exsheets-headings' Object}

\subsection{Load Custom Configurations}\secidx{Custom Configurations}
If you have custom configurations you want to be loaded automatically then save
them in a file \code{exsheets\_configurations.cfg}. If this file is present it
will be loaded \cmd*{AtBeginDocument}.
\secidx*{Custom Configurations}
\secidx*{\ExSheets}[ExSheets]
\SetupExSheets{headings=block}

\part{The \Tasks package (\tasksversion)}\label{part:tasks}
\secidx{\Tasks}[tasks]
\section{About the Documentation}
\noindent\sinceversion{0.9a}The \Tasks package has got its own documentation --
\url{tasks_en.pdf}.  Please read that file for details on the package.

\part{The \cntformats package (\cntfmtsversion)}
\label{part:cntformats}\secidx{\cntformats}[cntformats]
\section{About the Documentation}
\noindent\sinceversion{0.5}The \cntformats{} package has got its own documentation --
\url{cntformats_en.pdf}.  Please read that file for details on the package.

\appendix
\part{Appendix}
\addsec{A List of all Solutions used in this Manual}\label{sec:solutions:list}
\SetupExSheets{headings=block-wp,solutions-totoc}
\printsolutions

\RaggedRight\printbibliography

\setindexpreamble{Section titles are indicated \textbf{bold}, packages \textsf{sans
serif}, commands \code{\textbackslash\textcolor{code}{brown}}, options \textcolor
{key}{\code{yellow}} and modules \textcolor{module}{\code{blue}}.}
\printindex
\end{document}
