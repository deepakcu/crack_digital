%
% A simple LaTeX template for Books
%  (c) Aleksander Morgado <aleksander@es.gnu.org>
%  Released into public domain
%

\documentclass{book}
\usepackage[a4paper, top=3cm, bottom=3cm]{geometry}
\usepackage[latin1]{inputenc}
\usepackage{setspace}
\usepackage{fancyhdr}
\usepackage{tocloft}
\usepackage[]{exsheets}
\usepackage[]{enumitem}
\usepackage{listings}
\usepackage{multido}
\usepackage{circuitikz}


%\usepackage[totoc=true]{exsheets}



\begin{document}


\pagestyle{empty}
%\pagenumbering{}
% Set book title
\title{\textbf{Cracking the Digital Design Interview}}
% Include Author name and Copyright holder name
\author{Deepak Unnikrishnan}



% 1st page for the Title
%-------------------------------------------------------------------------------
\maketitle


% 2nd page, thanks message
%-------------------------------------------------------------------------------
\thispagestyle{empty}
\thanks{Thanks to Vishwas Vijayendra and other members of the Reconfigurable Computing Group, UMass Amherst\em{De finibus bonorum et malorum}}
\newpage



% General definitions for all Chapters
%-------------------------------------------------------------------------------

% Define Page style for all chapters
\pagestyle{fancy}
% Delete the current section for header and footer
\fancyhf{}
% Set custom header
\lhead[]{\thepage}
\rhead[\thepage]{}

% Set arabic (1,2,3...) page numbering
\pagenumbering{arabic}

% Set double spacing for the text
\doublespacing



% Not enumerated chapter
%-------------------------------------------------------------------------------
\chapter*{Preface}

This book is a collection of popular digital design interview questions. I was
inspired to compile the collection of problems presented in this book during
my final year as a graduate student. During this period, I was  interviewing several for full-time
positions at ASIC/FPGA companies. Soon, I found myself wondering how to
best prepare to crack the digital interview. I quickly realized that cracking
the interview requires not only a good academic background and work experience, but it also requires a good
foundation of fundamental digital design principles.
As a first step, myself and my friend, Vishwas Vijayendra started collecting interview questions from
popular Internet websites. We compiled these questions in a Google document and shared it internally
within our research group. As time went by, many members within and outside
our research group have contributed to our collection. To my surprise, I have found
many of these problems being reused across interviews in many companies.
From my experience as an interviewee, I have found that while having a good resume
can get you to the doors of the company, a good understanding of the fundamentals
is impeccable to cracking the interview. I would encourage readers to use these
problems to test your knowledge rather than using it as a comprehensive interview
preparation guide. Useful resources to brush up your fundamental knowledge in the
subject area have been provided towards the end of this book.

The collection of problems in this book have been compiled from a
variety of sources in the Internet. We don't own the copyright for these problems.
The copyright rests with the respective owners, when cited.
Credits are provided to the sources of these problems. Good luck!




% If the chapter ends in an odd page, you may want to skip having the page
%  number in the empty page
\newpage
\thispagestyle{empty}

% Last pages for ToC
%-------------------------------------------------------------------------------
\newpage
% Include dots between chapter name and page number
\renewcommand{\cftchapdotsep}{\cftdotsep}
%Finally, include the ToC
\tableofcontents


% First enumerated chapter
%-------------------------------------------------------------------------------

\chapter{Combinational Logic}

\begin{question}[ID=first]\label{qu:question_with_solution}
Using only NAND gates, NOR gates and 2:1 MUX design the following 2 input gates:
\begin{enumerate}[label=\alph*)]
\item AND
\item NOT
\item NOR
\item NAND
\item OR
\item XOR
\item XNOR
\item 2:1 MUX
\end{enumerate}
\end{question}
\begin{solution}[print]
Test sol
\begin{circuitikz} \draw 
(0,0) to[battery] (0,4)
      to[ammeter] (8,4) -- (4,0)
      to[lamp] (0,0)      
;
\end{circuitikz}
\end{solution}

\begin{question}[ID=first]\label{qu:question_with_solution}
Describe 2 ways of writing a 4:2 bit priority encoder in Verilog.
\end{question}
\begin{solution}
Test solution.

\end{solution}




\begin{question}[ID=first]\label{qu:question_with_solution}
How do you multiply a 8 bit value by 7?
\end{question}
\begin{solution}
Test solution.
\end{solution}

\begin{question}[ID=first]\label{qu:question_with_solution}
Design a 8:1 MUX using 2:1 MUX.
\end{question}
\begin{solution}
\end{solution}

\begin{question}[ID=first]\label{qu:question_with_solution}
How do realize inverter using 2 input NAND gate? Show me 2 ways of doing it.
\end{question}
\begin{solution}
Test solution.
\end{solution}







\printsolutions

\bibliography{references}






\end{document}
